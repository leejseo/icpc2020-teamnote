\subsection{Basic Number Theory} %ntopia
\inputminted{cpp}{src/Math/basic.cpp}

\subsection{Primality Test - Miller-Rabin} %ntopia
\inputminted{cpp}{src/Math/primality.cpp}

\subsection{Rational Number} %ntopia
\inputminted{cpp}{src/Math/rational.cpp}

\subsection{Pollard-Rho} %ntopia
\inputminted{cpp}{src/Math/pollard-rho.cpp}

\subsection{Chinese Remainder Theorem} %ntopia
\inputminted{cpp}{src/Math/crt.cpp}

%\subsection{Count Point Under the Line and Axes} % koosaga
%\inputminted{cpp}{src/Math/points-line.cpp}

%\subsection{Gaussian Elimination (real number)} %ntopia
%\inputminted{cpp}{src/Math/gaussian.cpp}

\subsection{Gaussian Elimination} %koosaga
\inputminted{cpp}{src/Math/gaussian-int.cpp}

\subsection{Linear Algebra} %koosaga
\inputminted{cpp}{src/Math/matrix.cpp}


\subsection{Score Theorem}

$D = (d_1, d_2, \cdots, d_n)$, $d_{i-1} \le d_i$가 그래프의 degree sequence가 될 수 있음은 $D' = (d_1', \cdots, d_{n-1}')$, $d_i' = \begin{cases} d_i & i < n-d_n \\ d_i-1 & i \ge n - d_n \end{cases}$
이 그래프의 degree sequence가 될 수 있음과 동치이다.

\subsection{Derangement}
$D_n$을 $[n] \to [n]$ 상의 순열 중 고정점이 없는 것의 갯수라고 하자.
다음의 식이 성립한다.
\begin{itemize}
    \item $D_n = (n-1)(D_{n-1}+D_{n-2}), \quad D_0 = 1,\, D_1 = 0.$
    \item $D_n = n! \sum_{k=0}^{n} \frac{(-1)^k}{k!}$
\end{itemize}

\subsection{Catalan Number}
\begin{itemize}
    \item $C_n = \frac{1}{n+1} {2n \choose n}$
    \item $C_n = \sum_{i=0}^{n-1} C_i C_{n-1-i}$
\end{itemize}

\subsection{Burnside's Lemma}
Let $X$ be a finite set and $G$ be a permutation group over $X$.
Define followings:
\begin{itemize}
    \item $G(X) := \{ \sigma(x) : \sigma \in G \}$ for each $x \in X$
    \item $X_\sigma := \{x \in X : \sigma(x) = x\}$ for each $\sigma \in G$
    \item $X/G := \{G(x) : x \in X \}$
\end{itemize}
Following holds.
$$ |X / G| = \frac{1}{|G|} \sum_{\sigma} |X_\sigma|. $$
\iffalse
\subsection{Laplace Matrix}
\begin{itemize}
    \item $q_{ii} = \deg (i)$
    \item $q_{ij} = \begin{cases} -1 & (i,j) \in E \\ 0 & \text{otherwise}\end{cases}$
\end{itemize}

$T(G) = |\det Q_{ij}|$.

\subsection{Berlekamp-Massey Algorithm}
\begin{itemize}
    \item \framebox{\texttt{berlekamp\_massey}} 함수. $n$개의 DP 결괏값이 주어졌을 때, Berlekamp-Massey 알고리즘을 사용하여 해당 DP의 점화식을 찾아주는 루틴. $O(n^2)$ 시간 복잡도에 작동한다 (정확히는, 답이 되는 점화식의 크기가 $k$ 일 때 $O(nk+n\log mod))$
    \item \framebox{\tt get\_nth} 함수. 주어진 DP 점화식을 사용하여 $n$번째 항을 구하는 루틴.
\end{itemize}
\inputminted{cpp}{src/Math/berlekamp.cpp}
\fi

%\subsection{Linear Sieve}
%\inputminted{cpp}{src/Math/linear-sieve.cpp} %ahgus89
%\begin{itemize}
%    \item $\tau(n)$: $n$의 양의 약수의 개수
%    \item $\sigma(n)$: $n$의 양의 약수의 합
%    \item $\phi(n)$: $n$ 이하의 $n$과 서로소인 수 개수
%    \item $\mu(n)$: $n$의 지수 2이상인 소인수가 있으면 0, 아니면 $(-1)^{\text{number of prime divisors}}$
%\end{itemize}

\subsection{Fast Fourier Transform}
\inputminted{cpp}{src/Math/fft.cpp} %myungwoo

\subsection{Number Theoretic Transform}
\inputminted{cpp}{src/Math/ntt.cpp} %cubelover

\subsection{Simplex Method}
\inputminted{cpp}{src/Math/simplex.cpp} %zigui / junis3

\subsection{Special Primes}
\begin{itemize}
\item $n \leq 1,000,000$ 약수 최대 240개 (720,720)
\item $n \leq 1,000,000,000$ 최대 1,344개 (735,134,400)
\item up to 10,000: 소수 1,229개 (9,973)
\item up to 100,000: 소수 9,592개 (99,991)
\item up to 1,000,000: 소수 78,498개 (999,983)
\item up to 1,000,000,000: 소수 50,847,534개 (999,999,937)
\item 10,007; 10,009; 10,111; 31,567; 70,001; 1,000,003; 1,000,033; 4,000,037
\item 99,999,989; 999,999,937; 1,000,000,007; 1,000,000,009; 9,999,999,967
\item $998244353 = 119 \times 2^{23} + 1$, primitive 3
\item $985661441 = 235 \times 2^{22} + 1$, primitive 3
\item $1012924417 = 483 \times 2^{21} + 1$, primitive 5
\end{itemize}